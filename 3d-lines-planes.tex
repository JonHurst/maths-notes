\documentclass[a5paper]{article}
\usepackage[utf8]{inputenc}
\usepackage[T1]{fontenc}
\usepackage{txfonts}
\usepackage{bm}
\usepackage{geometry}

\title{3d Lines and Planes}
\author{Jon Hurst}

\begin{document}
\maketitle
\section{Planes}

Any plane can be completely determined by a single position vector. The
direction of the vector is normal to the plane, which fixes its orientation, and
the length of the vector is the distance from the origin, which fixes its
position in space. To be more general, we can use a unit normal vector and a
length to achieve the same effect.

If $\bm{r}$ is the position vector of any point on the plane, then the length of
the projection of $\bm{r}$ on $\bm{\hat{n}}$ will be constant and equal to $d$,
the distance of the plane from the origin along the unit normal vector. Thus,
for any point on the plane:

\begin{equation}
  \bm{r} \cdot \bm{\hat{n}} = |\bm{r}|cos(\theta) = d \label{eq:1.1}
\end{equation}

If $\bm{n}$ is just any normal vector (i.e. not necessarily a unit vector) then
this is modified to:

\begin{equation}
  \bm{r} \cdot \bm{n} = |\bm{r}||\bm{n}|cos(\theta) = d|\bm{n}| = D \label{eq:1.2}
\end{equation}

where $D$ is a constant specific to the combination of the plane and the normal
vector that has been used. Note that (\ref{eq:1.2}) can be converted to
(\ref{eq:1.1}) simply by dividing through by $|\bm{n}|$.

To convert to cartesian form we substitute

\begin{eqnarray}
  \bm{r} = x\bm{i} + y\bm{j} + z\bm{k} \\
  \bm{n} = a\bm{i} + b\bm{j} + c\bm{k}
\end{eqnarray}

which gives

\begin{equation}
  ax + by + cz = D
\end{equation}

with $d$, the distance of the plane from the origin being

\begin{equation}
  d = \frac{D}{\sqrt{a^2 + b^2 + c^2}}
\end{equation}

Converting between vector and cartesian forms, then, is simply a matter of
reading the cartesian coefficients as the coefficients of the normal vector and
copying the $D$.

If you have two parallel planes, both with unit normal vector $\bm{\hat{n}}$
such that

\begin{eqnarray}
  \bm{r_1} \cdot \bm{\hat{n}} = d_1 \nonumber\\
  \bm{r_2} \cdot \bm{\hat{n}} = d_2 \nonumber
\end{eqnarray}

then the distance between them, $s$ is given by

\begin{equation}
  s\ =\ |d_1 - d_2|\ =\ |(\bm{r_1} - \bm{r_2}) \cdot \bm{\hat{n}})| \label{eq:1.3}
\end{equation}

This also allows the distance from a point to the plane to be easily found -- we
just find the distance to the parallel plane that contains the point.
Canonically, if we have a plane $ax + by + cz = D$ and a point $(p, q, r)$ then
the distance, $s$ between the point and the plane is

\begin{equation}
  s = \frac{|\ (p, q, r) \cdot (a, b, c) - D\ |}{\sqrt{a^2 + b^2 + c^2}}
\end{equation}

\section{Lines}

If the point $\bm{P}(x_0, y_0, z_0)$ has position vector $\bm{x_0} = x_0\bm{i} +
y_0\bm{j} + z_0\bm{k}$ and $\bm{a} = a\bm{i} + b\bm{j} + c\bm{k}$ is a direction
vector, then any point on the line parallel to $\bm{a}$ which passes through
$\bm{P}$ has position vector $\bm{r}$ where:

\begin{equation}
  \bm{r} = \bm{x_0} + \lambda\bm{a} \hspace{3em};\ \lambda\in\Re
\end{equation}

An alternative restriction comes from the fact that the magnitude of the cross
product is zero for parallel lines:

\begin{equation}
  (\bm{r} - \bm{x_0}) \times \bm{a} = \bm{0}
\end{equation}

The canonical cartesian form can be found by equating components and eliminating
$\lambda$. If $\bm{r} = x\bm{i} + y\bm{j} + z\bm{k}$ then:

\begin{equation}
  \frac{x - x_0}{a} = \frac{y - y_0}{b} = \frac{z - z_0}{c}
\end{equation}

Note how the that if we disregard the right-most equality we can convert what is
left into the form $y = mx + c$; i.e. the left-most equality defines the image
of the line on the plane $z = 0$ and the remaining equality provides the value
for $z$ at each point on that image.

A direction vector can easily be read off from the denominators of this standard
form. Mentally reintroducing $\lambda$ helps to remember how this is done.

The direction vector can be used to find ``direction cosines'', which are the
cosines of the angles between the direction vector and $\bm{i}$, $\bm{j}$ and
$\bm{k}$. For example, for the direction cosine of $\bm{a} = a\bm{i} + b\bm{j} +
c\bm{k}$ for $\bm{i}$:

\begin{equation}
  \bm{a} \cdot \bm{i} = |\bm{a}|cos(\theta_i) \Rightarrow cos(\theta_i) = \frac{a}{|\bm{a}|}
\end{equation}

The set of planes containing a single line is infinite since a line has a circle
of normal vectors, but this set is an infinitesimal subset of the set of all
planes. This is analogous to a line and an area both having an infinite number
of points, but the area having infinitely more. This means that it is not
necessarily possible to find a single plane that contains two lines.

A second line will, however, also have a circle of normal vectors, and if the
centre of the two circles are co-located they will intersect either in a single
line or, if the lines were parallel, be completely co-incident. If there is a
single line, then that line represents a \textit{common} normal vector, and thus
parallel planes can always be found that each contain one of the lines. This
allows us to calculate the minimum distance between two lines.

The cross product of the direction vectors of the lines is a common normal
vector, and each to the origin points is a point on their respective containing
parallel planes. This completely defines the planes, so we can use
(\ref{eq:1.3}) to find the distance between them.

If this distance between the parallel planes containing the lines happens to be
zero and the lines are not parallel, then the lines must intersect. Finding the
point of intersection is probably most easily done by finding the point of
intersection of the projections in a suitable plane, e.g $z = 0$, and then
checking whether the $z$ values are the same.

\end{document}
